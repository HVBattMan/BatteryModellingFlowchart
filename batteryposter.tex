\documentclass{beamer}
%% Possible paper sizes: a0, a0b, a1, a2, a3, a4.
%% Possible orientations: portrait, landscape
%% Font sizes can be changed using the scale option.
\usepackage[size=a0,orientation=portrait,scale=1.8]{beamerposter}
\usetheme{batteryposter}
\usecolortheme{batteryposter}

\usepackage[utf8]{inputenc}
\usepackage[T1]{fontenc}
\usepackage{libertine}
\usepackage[scaled=0.92]{inconsolata}
\usepackage[libertine]{newtxmath}
\usepackage{tikz}
\usepackage{lipsum}
\usepackage{standalone}
\usepackage{ragged2e}
% !TEX root = batteryposter.tex

\usepackage{mwe}
\usepackage{blindtext}
\usepackage{relsize}
\usetikzlibrary{positioning	}
\usetikzlibrary{arrows.meta}
\usetikzlibrary{calc}

%%%%%%%%%%%%%%%%%%%%%%%%%%%%%%%%%%%%%%%%%%%%%%%%%%%%%%%%%%%%%%%%%%%%%%%%%%%%%%%%%%%%%%%%
% Poster parameters %%%%%%%%%%%%%%%%%%%%%%%%%%%%%%%%%%%%%%%%%%%%%%%%%%%%%%%%%%%%%%%%%%%%
%%%%%%%%%%%%%%%%%%%%%%%%%%%%%%%%%%%%%%%%%%%%%%%%%%%%%%%%%%%%%%%%%%%%%%%%%%%%%%%%%%%%%%%%

% Colours
\newcommand{\macrocolor}{red}
\newcommand{\microcolor}{blue}
\newcommand{\macrotext}[1]{\color{\macrocolor}#1}
\newcommand{\microtext}[1]{\color{\microcolor}#1}
\newcommand{\black}[1]{\color{black}#1}

% Graph box parameters
\newcommand{\FIGWIDTH}{17cm} % figure width
\newcommand{\FIGHEIGHT}{12cm} % figure height
\newcommand{\VERTFIGSEP}{2cm} % vertical separation between figures
\newcommand{\HORFIGSEP}{4cm} % horizontal separation between figures
\tikzstyle{dimensions} = [anchor=north east]
\tikzstyle{physics} = [anchor=south east, align=left, font=\footnotesize]

% Axis box and line style
\tikzstyle{axisbox} = [inner sep=0]
\tikzstyle{axisline} = [
		% {Latex[length=1cm, width=1cm]}-{Latex[length=1cm, width=1cm]},
		-{Latex[length=1cm, width=1cm]},
		ultra thick,
		>=stealth
]

% Model point style
\tikzstyle{modelpoint} = [
		circle,
		fill=black,
		inner sep=0pt,
		minimum size=0.5cm,
		% prefix after command= {\pgfextra{\tikzset{every label/.style={font=\Huge}}}}
]
\tikzstyle{results_arrow} = [
		-{Latex[length=1cm, width=1cm]},
		ultra thick,
		>=stealth
]

% Textbox style
\tikzstyle{textbox} = [
		draw=red,
		fill=blue!20,
		very thick,
    rectangle,
		rounded corners,
		inner sep=10pt,
		inner ysep=20pt]
\tikzstyle{textboxtitle} = [fill=red, text=white]

% Cross-dimensions
\newcommand{\xo}{\bigotimes}
\newcommand{\includedimensions}[1]{}
%%%%%%%%%%%%%%%%%%%%%%%%%%%%%%%%%%%%%%%%%%%%%%%%%%%%%%%%%%%%%%%%%%%%%%%%%%%%%%%%%%%%%%%%
%%%%%%%%%%%%%%%%%%%%%%%%%%%%%%%%%%%%%%%%%%%%%%%%%%%%%%%%%%%%%%%%%%%%%%%%%%%%%%%%%%%%%%%%

\usepackage[utf8]{inputenc}
\usepackage[T1]{fontenc}                                                                           
\usepackage{libertine}  
\usepackage[libertine]{newtxmath}

\usepackage{shellesc}

\usepackage{siunitx}


\usepackage{pgfplots}
%\usepackage{pgfplotstable}
\usepgfplotslibrary{colormaps} 
%\usepackage[caption=false]{subfig}
\usepackage{subcaption} 
\usepackage{graphicx, floatrow}

\usepackage{lipsum}  

\usetikzlibrary{backgrounds}

%
%usepackage{pgf}
%usepackage{mathtools}
\pgfplotsset{compat=newest}

%\usepackage{tikz}
%\usetikzlibrary{external}
%\tikzexternalize % activate!

\definecolor{c1}{RGB}{228,26,28}
\definecolor{c2}{RGB}{55,126,184}
\definecolor{c3}{RGB}{77,175,74}
\definecolor{c4}{RGB}{152,78,163}
\definecolor{c5}{RGB}{255,127,0}
\definecolor{c6}{RGB}{106,61,154}



%\usefonttheme[onlymath]{serif}


\newcommand{\PLOTLINES}{0.7mm}
\newcommand{\AXISLINEWIDTH}{0.7mm}


\pgfplotscreateplotcyclelist{mylist}{
{c1, line width=\PLOTLINES}, 
{c2,line width=\PLOTLINES},
{c3,line width=\PLOTLINES},
{c4,line width=\PLOTLINES},
{c5,line width=\PLOTLINES},
{c1,dashed,line width=\PLOTLINES},
{c1,dotted,line width=\PLOTLINES},
{c2,dashed,line width=\PLOTLINES},
{c2,dotted,line width=\PLOTLINES},
{c3,dashed,line width=\PLOTLINES},
{c3,dotted,line width=\PLOTLINES},
{c4,dashed,line width=\PLOTLINES},
{c4,dotted,line width=\PLOTLINES},
{c5,dashed,line width=\PLOTLINES},
{c5,dotted,line width=\PLOTLINES}}


\newenvironment{customlegend}[1][]{%
    \begingroup
    % inits/clears the lists (which might be populated from previous
    % axes):
    \csname pgfplots@init@cleared@structures\endcsname
    \pgfplotsset{#1}%
}{%
    % draws the legend:
    \csname pgfplots@createlegend\endcsname
    \endgroup
}%

% makes \addlegendimage available (typically only available within an
% axis environment):
\def\addlegendimage{\csname pgfplots@addlegendimage\endcsname}


%
\author[]{
  	Jon Chapman,
  	Matt Hennessy,
  	Toby Kirk,
	Scott Marquis,\\
  	Colin Please,
  	Ian Roper,
	Valentin Sulzer,
	Robert Timms.
}
\title{An Asymptotic Framework\\ for Battery Modelling}
\institute{Your Institution}
% Optional foot image

\begin{document}
\begin{frame}[fragile]
\centering

\vspace{-2em}
\begin{columns}
  \begin{column}{0.9\linewidth}
	\begin{topblock}{}
	  \small
%Electrochemical and equivalent-circuit modelling are the two most popular approaches to battery simulation, but the former is computationally expensive and the latter provides limited physical insight. A theoretical middle ground would be useful to support battery management, on-line diagnostics, and cell design. Through asymptotic analysis we can systematically derive simplified models which have their roots in physics-based models, but offer reduced computational complexity. Combinations of simplifications to the {\macrotext{macroscale}}  and {\microtext{microscale}} result in a suite of reduced-order models, whose complexity and fidelity can be selected to best meet the user's needs. The approach taken presents a general framework for developing reduced-order models which is independent of chemistry, and can be extended to incorporate additional physical effects such as mechanics, side reactions and degredation mechanisms. We are currently developing the software package PyBaMM, in collaboration with the Faraday Institution, which numerically implements the suite of models which lie within the common modelling framework.

Electrochemical and equivalent-circuit modelling are the two most popular approaches to battery simulation, but the former is computationally expensive and the latter provides limited physical insight. Through asymptotic analysis, we systematically derive simplified physics-based models, which provide a useful theoretical middle ground to support battery management, on-line diagnostics, and cell design. Combinations of simplifications to the {\macrotext{macroscale}} and {\microtext{microscale}} result in a suite of reduced-order models, whose complexity and fidelity can be selected to best meet the user's needs. The approach provides a general framework for developing reduced-order models which is independent of chemistry, and can be extended to incorporate additional physical effects such as mechanics, side reactions, and degredation mechanisms. We are currently developing the software package PyBaMM to implement the suite of models numerically.


\end{topblock}
  \end{column}
\end{columns}



% \textcolor{white}{\rule{0.9\textwidth}{0.1em}}
\vskip0.6em


% Tikz picture for graphbox

\begin{columns}
  \begin{column}{0.91\linewidth}
	\centering
	\includestandalone[scale=0.93]{graph_box}
  \end{column}
\end{columns}

\vspace{-1mm}
% Can someone write in the references properly
\begin{columns}
  \begin{column}{0.91\linewidth}
	\begin{block}{References}
	  \tiny
	[1] V. Sulzer, S.J. Chapman, C.P. Please, D.A. Howey, and C.W. Monroe. Faster
	Lead-Acid Battery Simulations from
	Porous-Electrode Theory: II. Asymptotic
	Analysis. \emph{Submitted}, 2019. \hspace{0.5em} \emph{arXiv preprint} arXiv:1902.01774 \\

	[2] I.R. Moyles, M.G. Hennessy, T.G. Myers, and B.R. Wetton. Asymptotic reduction of a 	porous electrode model for lithium-ion batteries. \emph{Submitted}, 2018. \hspace{0.5em}  \emph{arXiv preprint} arXiv:1805.07093v2
	\end{block}
  \end{column}
\end{columns}

\end{frame}
\end{document}
